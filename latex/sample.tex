% 1  one inch + \hoffset      2  one inch + \voffset
% 3  \oddsidemargin = 22pt    4  \topmargin = 22pt
%   or \evensidemargin
% 5  \headheight = 13pt       6  \headsep = 19pt
% 7  \textheight = 595pt      8  \textwidth = 360pt
% 9  \marginparsep = 7pt     10  \marginparwidth = 106pt
% 11 \footskip = 27pt            \marginparpush = 5pt (not shown)
%    \hoffset = 0pt              \voffset = 0pt
%    \paperwidth = 597pt         \paperheight = 845pt

%\setlength{\oddsidemargin}{0.5cm}
%\setlength{\evensidemargin}{0.5cm}
%\setlength{\marginparsep}{0.75cm}
%\setlength{\marginparwidth}{2.5cm}
%\setlength{\marginparpush}{1.0cm}
%\setlength{\textwidth}{150mm}

%% \documentclass[a4paper,10pt]{article}
\documentclass{article}
\usepackage[backend=biber]{biblatex}
\addbibresource{refs.bib}

\usepackage{listings}
\usepackage{amsmath,amsthm,amssymb}

\usepackage[T1]{fontenc}
\usepackage[utf8]{inputenc}
\usepackage[brazil]{babel}

\theoremstyle{definition}
\newtheorem{theorem}{Teorema}[section]
\newtheorem{definition}{Definição}[section]
\newtheorem{lemma}[theorem]{Lema}
\newtheorem{proposition}[theorem]{Proposição}
\newtheorem{corollary}[theorem]{Corolário}
\newtheorem{example}{Exemplo}[section]
\newtheorem{remark}{Comentário}[section]

\newtheoremstyle{pf}        % name
                {10pt}      % space above
                {3pt}       % space below
                {}          % body font
                {}          % indent amount (empty = 0, \parindent = para)
                {\itshape}  % head font
                {.}         % punctuation after head
                {.5em}      % space after head
                            % \newline = linebreak
                {}          % head spec (empty means `normal')

\theoremstyle{pf}
\newtheorem*{pf}{Prova}

% references
\newcommand{\reflema}[1]{Lema \ref{#1}}
\newcommand{\refth}[1]{Teorema \ref{#1}}
\newcommand{\refex}[1]{Exemplo \ref{#1}}
\newcommand{\refco}[1]{Corolario \ref{#1}}
\newcommand{\refdef}[1]{Definição \ref{#1}}

% sets
\newcommand{\reals}{\mathbb{R}}
\newcommand{\ints}{\mathbb{Z}}
\newcommand{\rats}{\mathbb{Q}}
\newcommand{\nats}{\mathbb{N}}

% tags
\newcommand{\de}[1]{\jmp\begin{definition}[#1]}
\newcommand{\ed}{\end{definition}}
\newcommand{\thm}[1]{\jmp\begin{theorem}[#1]}
\newcommand{\mht}{\end{theorem}}
\newcommand{\rem}[1]{\jmp\begin{remark}[#1]}
\newcommand{\mer}{\end{remark}}
\newcommand{\prf}{\begin{pf}}
\newcommand{\frp}{\qed\end{pf}}
\newcommand{\ex}[1]{\jmp\begin{example}[#1]}
\newcommand{\xe}{\end{example}}

% mascii tags
\newcommand{\masciidef}[1]{\jmp\begin{definition}[#1]}
\newcommand{\defend}{\end{definition}}
\newcommand{\masciith}[1]{\jmp\begin{theorem}[#1]}
\newcommand{\theend}{\end{theorem}}
\newcommand{\masciiproof}{\begin{pf}}
\newcommand{\profend}{\qed\end{pf}}
\newcommand{\masciiexe}[1]{\jmp\begin{exercise}[#1]}
\newcommand{\exeend}{\end{exercise}}
\newcommand{\masciicor}[1]{\jmp\begin{corollary}[#1]}
\newcommand{\corend}{\jmp\end{corollary}}
\newcommand{\masciirem}[1]{\jmp\begin{remark}[#1]}
\newcommand{\remend}{\end{remark}}
\newcommand{\masciilemma}[1]{\jmp\begin{lemma}[#1]}
\newcommand{\lemend}{\end{lemma}}
\newcommand{\masciiprop}[1]{\jmp\begin{proposition}[#1]}
\newcommand{\propend}{\end{proposition}}

% jumps
\newcommand{\jmp}{\vspace{3pt}}

% shortcuts
\newcommand{\pit}{{\em Prova. }}
\newcommand{\normal}{\unlhd}
\newcommand{\cyclic}[1]{\langle #1 \rangle}
\newcommand{\imp}{$\implies$}

% mathematical functions
\DeclareMathOperator{\mdc}{mdc}
%% \DeclareMathOperator{\floor}{floor}
\newcommand{\floor}[1]{\lfloor #1 \rfloor}

\author{Fulano de Tal\\
        \tt{fulano@example.bit}}
\title{Relatório 1}
\date{11 de setembro de 2017}

\begin{document} \maketitle

%% \abstract{Relatório 1 de observações feitas durante estudo orientado
%%   em tópicos da teoria dos números com ênfase em criptografia,
%%   semestre 2017.2.  Neste breve relatório ênfase foi dada ao Teorema de
%%   Lagrange e breve seleção de corolários relativos à busca por
%%   subgrupos.}

\lstset{
  basicstyle=\small
}

\setlength{\parskip}{7pt}
\setlength{\parindent}{0pt}

%% Pra bookmarcar algum ponto, use \label{tag}.  Pra fazer referência
%% a um label, use \ref{tag}.  Pra citar publicações em refs.bib, use
%% \cite[veja página 18]{ss}, onde ss é uma tag em refs.bib.  No meu
%% caso aqui, ss é de [s]alahoddin [s]hokranian.  Por exemplo, eis uma
%% citação\cite[página 1]{maxfield}.  Agora uma do Menezes\cite[página
%% 287]{hac}.  Pronto pra começar.

\section{Sobre o teorema de Lagrange}

O Teorema de Lagrange nos garante que se existir um subgrupo $H$ de
$G$, seu tamanho é de um divisor do tamanho de $G$, embora não nos
garanta que haverá tal grupo.

\begin{lemma} \label{lemalagrange} Seja $G$ um grupo, $H$ um subgrupo 
de $G$. Cada classe lateral à direita tem a mesma cardinalidade que
$H$. Em particular, quaisquer duas classes laterais à direita têm a
mesma cardinalidade uma da outra.
\end{lemma} 

\prf Seja $Hg$ uma classe lateral à direita. Defina $p:H \to Hg$ por
$p(h) = hg$. Preciso mostrar que $p$ é injetora, o que ela é porque
$p(h_1) = p(h_2)$ implica $h_1 g = h_2g$ e daí obtemos $h_1 g g^-1=
h_2 g g^{-1}$ e portanto $h_1 = h_2$.  Precisamos mostrar que $p$ é
sobrejetora.  Pegue um $y$ em $Hg$ e por isso $y = hg$ para algum $h
\in H$.  Agora note que $p(h) = hg = y$. Isso mostra que $p$ é
sobrejetora e portanto bijetora.  Isso implica que a cardinalidade de
$H$ é a mesma de $Hg$, isto é $|H| = |Hg|$.\frp

Isso nos permite provar o Teorema de Lagrange com facilidade.  Note
que embora o lema acima não se restrinja a grupos finitos, o Teorema
de Lagrange o faz.

\thm{Lagrange} \label{lagrange} Se $H$ é subgrupo de $G$, um grupo
finito, então a ordem de $H$ divide a ordem de $G$.\mht

\prf Considere $G/H.$ A operação binária de $G/H$ é uma classe de
equivalência e portanto constrói conjuntos disjuntos $S_1, ..., S_k$
tal que a união de todos nos dá o grupo $G$. Como a união de todos os
elementos dos membros de $G/H$ é exatamente G, então tem ser o caso
que $|G| = |S_1| + ... + |S_k|$.  Pelo \reflema{lemalagrange}, $|S_i|
= |H|.$ Então %align by =
%v
\begin{align*}
        |G| &= |S_1| + ... + |S_k|\\
            &= k |H|,
\end{align*}
%e
como desejado.\frp

Como aplicação do \refth{lagrange}, vejamos dois teoremas que me
parecem úteis no âmbito do problema do logaritmo discreto.

\thm{} Se $g^t = e$, então $t$ divide $|G|$. \mht

\prf Pelo \refth{lagrange}, a ordem do subgrupo cíclico gerado por $g$
divide a ordem de $G.$ Mas a ordem do subgrupo gerado por $g$ é igual
à ordem de $g.$ Logo, a ordem de $g$ divide a ordem de $G$\cite[página
  32, primeiro parágrafo]{scc}.\frp

\thm{} Todo grupo de ordem prima é cíclico e qualquer elemento 
não-neutro de $G$ é um gerador. \mht 

\prf Escolha $g \neq e \in G.$ Considere $\cyclic{g}.$ Pelo
\refth{lagrange}, a ordem de $\cyclic{g}$ precisa dividir $|G|,$ que é
$p,$ um número primo. Logo, $\cyclic{g}$ tem ordem $1$ ou $p.$ Mas
$\cyclic{g}$ contém pelo menos dois elementos, $e$ e $g.$ Logo
$\cyclic{g}$ não pode ter ordem $1$ e logo deve ter ordem $p$ e
$\cyclic{g} = G.$ \frp

%% \section{Sobre ideais}

%% Comecemos com exemplos.  

%% \begin{verbatim}
%% Conheço alguns exemplos de anéis --- Z, Q, R, Z(n), {0} ---, mas não
%% conheço exemplos de ideais.  Além de {0}, que é muito trivial, qual o
%% primeiro ideal a se pensar?

%% (*) Exemplo

%% O inteiros pares, 2Z, formam um ideal em Z.  (Veja também nZ.)

%% (*) Exemplo

%% Os polinômios, de coeficientes reais, divisíveis por x^2 + 1 formam um
%% ideal em R[x].

%% (*) Exemplo

%% O conjunto de matrizes n-por-n cuja última linha é zero forma um ideal
%% de matrizes n-por-n.

%% --8<---------------cut here---------------start------------->8---
%% Forma um ideal pela direita.  Numa multiplicação entre i em I e a em A,
%% o ideal pela direita coloca seu elemento na *esquerda*.  Ou seja, a
%% matriz cuja última linha é zero posiciona-se do lado esquerdo da
%% multiplicação.

%%   i*a em I

%% Assim, 

%%   [1 1] [a b] = [a + c  b + d]
%%   [0 0] [c d]   [  0      0  ]

%% Ou seja, em um right-ideal, o elemento do ideal fica na esquerda.
%% --8<---------------cut here---------------end--------------->8---

%% (*) Exemplo

%% O anél C(R) de funções contínuas contém o ideal de funções f tal que
%% f(1) = 0.

%% --8<---------------cut here---------------start------------->8---
%% Esse exemplo é interessante para conjecturarmos alguma intuição sobre
%%  ideais.

%% Seja g(x) contínua e f(x) em I.  Considere f(1)g(1).

%%    g(1) f(1) = g(1)*0 = 0 em I.
%% --8<---------------cut here---------------end--------------->8---
%% \end{verbatim}

%% \section{Sobre o anel $Z_n[x]$}

%% \begin{verbatim}
%% --8<---------------cut here---------------start------------->8---
%% Em primeiro lugar, Z(n) é o conjunto {0, 1, ..., n-1}, isto é, z mod n.
%% Então Z(n)[x] é o anel de polinômios cujos coeficientes são elementos de
%% Z(n).

%% O propósito aqui é mostrar que o subgrupo aditivo <x> é bem diferente do
%% ideal (x), o ideal gerado a partir do polinômio x.

%%   <x,+> = {x, 2x, 0}

%% porque 2x + x = 3x e 3 = 0.  Já

%%   (x) = {0, x, 2x, 2x^2, x^3, ...} = {f(x) : f(0) = 0} 
%% --8<---------------cut here---------------end--------------->8---
%% \end{verbatim}

\section{Sobre o número de dígitos em um inteiro}

O exemplo 2.51\cite[capítulo 2, página 58]{hac} nos informa que o
número de bits em um inteiro é $1 + \floor{\lg n}$.  Shokranian nos
ensina a calcular o número de algarismos\footnote{``Algarismo'' vem de
  al-Khowarazmi, que significa ``o homem de Khwarazm.''  É uma
  referência a Ja'far Mohammed Ben Musa, que era conhecido como o
  homem de Khwarazm.  O termo ``algoritmo'' também decorre dele.  Por
  volta do ano 825, al-Khowarazmi escreve um livro sobre aritmética
  explicando como usar os numerais do sistema Hindu-Arábico.  Mais
  tarde o livro foi traduzido e apareceu em latim sob o título ``Liber
  Algorismi'', que significa ``o livro de al-Khowarazmi.''\cite[página
    21]{eti}} em um inteiro na base 10\cite[página 13]{ss}, mas a
prova do teorema é diretamente generalizável a qualquer base, então
trocamos a base 10 por uma base $M$.

\thm{} \label{nbits} Em base $M,$ o número de dígitos de um inteiro $P$ é 
%
   $$1 + \floor{\log_M P}$$
%
\mht
\prf Escreva $P$ em potências de base $M$.  (Por exemplo, em base
10, $123 = 3*10^0 + 2*10^1 + 1*10^2$.)
%
   $$P = d_{n-1}M^{n-1} + d_{n-2}M^{n-2} + ... + d_{1}M^1 + d_{0}M^0$$
%
onde $d_i$ é o $i$-ésimo dígito de $P$.  Agora expresse $P$ colocando
$M^{n-1}$ em evidência.
%
   $$P = M^{n-1} (d_{n-1} + d_{n-2}/M + ... + d_{1}/M^{n-2} + d_{0}/M^{n-1})$$
%
Seja $B = d_{n-1} + d_{n-2}/M + ... + d_{1}/M^{n-2} + d_{0}/M^{n-1}$ de
tal forma que 
%
   $$P = M^{n-1} B$$
%
Note que cada cada $d_i < M$ porque cada $d_i$ é um dígito em base
$M$.  Então $B < M$ porque note que $d_{n-1}$ é um inteiro menor que
$M$ e então $d_{n-1}$ tem que estar distante de $M$ em pelo menos uma
unidade.  O resto da soma nunca chega a totalizar uma unidade porque
cada termo é dividido por uma potência de $M$.  Também note que 
$B < M$ implica $\log B < 1$.

Computando o log da equação acima, obtemos
%
\begin{align}
   \log P &= (n - 1) \log M + \log B\\
          &= (n - 1) + \log B
\end{align}
%
Então $n - 1 = \log P - \log B$.  Uma vez que $n - 1$ é um intiro e
$\log B < 1$, essa subtração não retira mais do que uma unidade.  Então
%
\begin{align}
    n - 1 &= \floor{\log P}\\
        n &= 1 + \floor{\log P}
\end{align}
%
O que $n$ representa é o número de algarismos usados para representar
$P$ em base $M$, como desejado.
\frp

\rem{} A equação 
%
  $$n - 1 = \log P - \log B$$
%
é curiosa porque temos um inteiro equacionado a uma diferença de
logaritmos.  Temos que notar $B$ é cuidadosamente construída a partir
de $P$.  $B$ é $P$ dividido por $M^{n - 1}$.  Então em base 10 por
exemplo, seja $P = 87849$ com 5 dígitos e então $B = 8.7849$ porque
$M^{n - 1} = 10^4$.  Agora, $\log P = 4.9437...$ enquanto que $\log B
= 0.9437...$ A mesma expansão decimal na parte fracionada.  Subtraindo
$\log B$ de $\log P$, obtemos o inteiro 4.\mer

\ex{} Eis uma aplicação desse conhecimento.  Digamos que nossos
computadores sejam capazes de computar com números tão grandes quanto
$2^{64}$.  Quão grande é isso?  Poderíamos expressar $2^{64}$ em
algarismos da base 10 para saber quantos algarismos são necessários.
Mas se só queremos saber quantos algarismos são necessários, podemos
aplicar o teorema acima: $1 + \floor{\log_{10} 2^{64}} = 1 + 19 = 20$
algarismos.  Similarmente se quisermos saber o número de bits
necessários para representar um certo inteiro, calculamos
$\floor{\lg n} + 1$, como informado por Menezes.  \xe

%% \section{Erro de digitação\cite[página 17, 18]{scc}}

%% Quando ele diz onde $k=1,2,...,\floor{n/k}$, ele quer dizer onde
%% $t=1,2,...,\floor{n/k}$ --- mera troca de letra.

%% Em ``[c]omo $\lg(n) \leq \sqrt{n}$ se $n \geq 4$, então o custo do
%% crivo de Eratóstenes é inferior a $O(n^2)$'', precisamos restringir
%% que $n \geq 2^4$ e não apenas $n \geq 4$.

\section{Sobre desertos de primos}

Eis um argumento elementar para se obter um deserto de primos de
tamanho arbitrário. Digamos que você queira um deserto de $4$ números.
Comece com $(4+1)!$ e compute:%align by =
%v
\begin{align*}
        (4+1)! &= 5! \\
        &= 2*3*4*5 \\
        &= 120.
\end{align*}
%e
Você sabe que $120 + 2 = 122$ não é primo porque $2$ é fator de $120$
e portanto $122$ também é divisível por $2.$ Similarmente, você sabe
que $123$ não é primo porque $120$ é divisível por $3$ e portanto
$123$ também.  Similarmente, $124$ é composto porque $4$ o
divide. Similarmente, $5$ divide $125.$ Logo, entre $122$ e $125$ não
há primo algum. Aí está seu deserto de primos em $4$ números.

Isso nos dá uma técnica de encontrar um deserto tão grande quanto 
queiramos. Se você quer um deserto de $560$ números, então comece com 
%v
        $$M = 561! = 2*3*4*5*6*...*561$$
%e
Certamente $M+2$ é composto pois $2$ o divide. Similarmente, $M+3$ é
composto pois $3$ o divide. Similarmente, $M+561$ é composto pois
$561$ o divide. Logo, de $561!+ 2$ até $561! + 561$ não há qualquer
primo.

%% \subsection{O lema ``2n escolhe n''}

%% Seja n um inteiro positivo.  Então 
%% %
%%   $$C(2n, n) = \Pi_{p \in P(2n)} p^{k_p}$$
%% %

%% Estudemos esse lema.

%% Estudemos também $2^n <= (2n^2)^\#$.

%% O quociente de dois primoriais nos dá o produto de primos
%% consecutivos.

\section{Sobre complexidade}

Em análise de algoritmos, usualmente não nos preocupamos com o custo
de operações primitivas (relativa a um computador) como a
multiplicação, divisão, soma e subtração de inteiros, mas no estudo de
algoritmos da teoria dos números, nos preocupamos com a computação de
números grandes, onde o custo dessas operações aritméticas se torna
relevante.  Por exemplo, no problema de ordenação estamos preocupados
com quantos números precisamos ordenar, não nos preocupando com o
tamanho deles.  Já no problema da fatoração, é obrigatório considerar
o tamanho do número a ser fatorado.  Por isso precisamos ajustar nosso
raciocínio sobre o custo dessas operações.

Se medirmos o tamanho dos números da entrada pela quantidade de bits,
quando teremos muitas chances de aplicar o \refth{nbits}, então
diremos que um algoritmo é eficiente se ele satisfizer a seguinte
definição.

\de{} \label{polytime} Um procedimento cuja entrada são os inteiros
$a_1, a_2, ..., a_k$ tem custo polinomial se ele termina em tempo
polinomial relativo a $$\lg(a_1), \lg(a_2), ...., \lg(a_k),$$
respectivamente, isto é, relativo ao tamanho da entrada.\ed

Em contextos onde o tamanho dos números é irrelevante, uma função que
multiplica dois inteiros $a$, $b$ tem custo $\Theta(1)$, pois consome
apenas uma multiplicação. No nosso contexto, o custo da multiplicação
--- que é de ordem quadrática --- é contabilizado, então o custo é
$\Theta(\max(\lg a,\lg b)^2)$.  

\ex{} Por exemplo, considere o problema de multiplicar inteiros da
ordem de $2^{161}$. Num determinado computador, o custo no pior caso
será de $(161+1)^2 = 162^2 = 26244$ operações porque a multiplicação
tem custo quadrático. Se aumentamos o tamanho da entrada em apenas um
bit, o custo sobe para $163^2 = 26569$ operações. Não é apenas uma
operação a mais. São 325 operações a mais.\xe

\ex{} Considere um problema cuja solução tem custo exponencial em base
2. Se a entrada tem tamanho 61 bits, o custo no pior caso é de
2305843009213693952 operações. Se aumentamos a entrada para 62 bits, o
custo passa a ser 4611686018427387904. São 2305843009213693952
operações a mais. Um bit a mais na entrada dobra a quantidade de
operações. É um crescimento relevante do volume de trabalho.\xe

\ex{} Considere o problema de testar se um número $n$ é primo. Uma
forma é verificar se algum número da lista $2, 3,$ ..., $\sqrt{n}$
divide $n.$ Se não dividir, $n$ é primo. Digamos inicialmente que o
custo desse procedimento é $\sqrt{n} - 1$ divisões, que é um custo
sublinear. Mas refaçamos a análise com base na quantidade de bits como
entrada. A entrada é um número $n,$ que ocupa uma quantidade da ordem
de $\lg n$ bits. O custo do procedimento é da ordem de $\sqrt{n}.$
Veja que o custo não está expresso em $\lg n$ bits, mas sim relativo à
$n.$ Quanto é $n$ em termos de $\lg n$ bits? Seja $x = \lg n.$ Assim
$n = 2^x.$ Logo, o custo é da ordem de $\sqrt{2^x} = 2^{x/2}$ relativo
à quantidade de bits da entrada.  (A propósito, levando em
consideração que cada divisão é da ordem de $\lg^2 x$, o custo é da
ordem $2^{x/2} \lg^2 x$, que é da mesma ordem que $2^{x/2}$.)\xe

\rem{} É por isso que se usa a expressão ``exponencial em $\lg n$.''
O que se quer dizer é que o custo é expresso com respeito a uma certa
quantia.  É por isso que na \refdef{polytime}, a variável independente
é o logaritmo na base 2 dos números arbitrários fornecidos pela
entrada.  Qualquer análise matemática é relativa a um conjunto de
primitivas. \mer

\ex{} No cálculo do custo do crivo de Eratóstenes\cite[página 17,
  18]{scc}, chega-se ao fato de que o custo é inferior a $O(n^2)$.  O
custo todavia não é quadrático.  O custo é exponencial porque relativo
à $\lg n$, $n^2 = 2^{2x}.$\xe

\printbibliography 
 
\end{document}


